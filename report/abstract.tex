With the current trend in cloud computing and virtualization, more organizations are moving their systems from a physical host to a virtual server.  Although this can significantly reduce hardware, power, and administration costs, it can increase the cost of analyzing performance problems.  With virtualization, there is an initial performance overhead, and as more virtual machines are added to a physical host the interference increases between various guest machines.   When this interference occurs, a virtualized guest application may not perform as expected.   There is little or no information to the virtual OS about the interference, and the current performance tools in the guest are unable to show this interference.

We examine the interference that has been shown in previous research, and relate that to existing tools and research in root cause analysis.  We show that in virtualization there are additional layers which need to be analyzed, and design a framework to determine if degradation is occurring from an external virtualization layer.  Additionally, we build a virtualization test suite with Xen and PostgreSQL and run multiple tests to create I/O interference.  We show that our method can distinguish between a problem caused by interference from external systems  and a problem from within the virtual guest.
 
